% preambule (fold)
\documentclass{article}

\usepackage[french]{babel}
\usepackage{endnotes}
\usepackage{enumerate}
\usepackage{graphicx}
\usepackage[pdfborder={0 0 0}]{hyperref}
\usepackage{subfig}

% font definitions (fold)
\usepackage{fontspec}
\usepackage{xunicode}
\usepackage{xltxtra}
\defaultfontfeatures{Numbers=OldStyle, Ligatures={Common}, Mapping=tex-text} \setromanfont[ItalicFont={* Slanted}, BoldFont={* Demi}, SmallCapsFont={* Caps}] {Latin Modern Roman}
\setsansfont{Latin Modern Sans}
\setmonofont[Scale=MatchLowercase]{Monaco}
\newfontfamily\zapfino{Zapfino}
\newfontfamily\typewriter{American Typewriter}
% font definitions (end)
% custom commands and environment (fold)
\newcommand{\footlink}[1] {\endnote{\href{#1}{\texttt{#1}}}}
\newcommand{\ie} {c.-à-d. }
\newcommand{\eg} {p.~ex. }
\newenvironment{resumeChapitre}
  {\begin{center}\small\bf\begin{em}}
  {\end{em}\end{center}}
% custom commands and environment (end)
% filigrane (fold)
\usepackage{eso-pic,xcolor}
\AddToShipoutPicture {
  \unitlength=1in
  \put(1.55,3.5) {
    \rotatebox{45} {
      \textcolor{lightgray} {
        \fontsize{48}{48}\bf\typewriter COPIE DE TRAVAIL
      }
    }
  }
}
% filigrane (end)
% captions pour listings (fold)
\usepackage{caption}
\usepackage{color}
\DeclareCaptionFont{white} {
  \color{white}\sffamily
}
\DeclareCaptionFormat{listing} {
  \colorbox[cmyk]{0.43, 0.35, 0.35, 0.01} {
    \parbox{\textwidth-17pt} {
      \hspace{7pt}#1#2#3
    }
  }
}
\captionsetup[lstlisting] {
  format=listing,
  labelfont=white,
  textfont=white,
  singlelinecheck=false,
  margin=0pt,
  font={bf,footnotesize}
}
% captions pour listings (end)
% code environment declaration (fold)
\usepackage{listings}
\lstnewenvironment{snippet}[1][] {
  \lstset {
    basicstyle=\small\ttfamily,
    stringstyle=\small\ttfamily,
    showstringspaces=false, #1
  }
}{}
\lstnewenvironment{code}[1][] {
  \lstset {
    basicstyle=\small\ttfamily,
    xleftmargin=17pt,
    framexleftmargin=14pt,
    framexrightmargin=-3pt,
    framexbottommargin=4pt,
    frame=b, #1
  }
}{}
% \begin{code}[gobble=2,label=helloworld,caption=Bonjour le monde]
%   println "Hello World"
%   // ...
% \end{code}
% code environment declaration (end)
% césures de mots (fold)
\lccode`\'="2019\lccode"2019="2019 % césure de mots avec un apostrophe
\hyphenation{Java-Script immu-able l'in-for-ma-tion Internet-Explorer}
% césures de mots (end)
% preambule (end)

\begin{document}

% title (fold)
\title{}
\author{François-Xavier Guillemette}
\maketitle
% title (end)

\begin{abstract}

\end{abstract}

\pagebreak
\tableofcontents
\pagebreak

\section{Technologies utilisées} % (fold)
\label{sec:technologies_utilisees}

\subsection{JAX-RS} % (fold)
\label{sub:jax_rs}
% subsection jax_rs (end)

\subsection{Groovy, JavaScript, Java} % (fold)
\label{sub:groovy}
% subsection groovy (end)

\subsection{Spring Framework} % (fold)
\label{sub:spring_framework}
% subsection spring_framework (end)

\subsection{Quartz} % (fold)
\label{sub:quartz}
% subsection quartz (end)

\subsection{JMX} % (fold)
\label{sub:jmx}
% subsection jmx (end)

\subsection{Hibernate} % (fold)
\label{sub:hibernate}
% subsection hibernate (end)

\subsection{PostgreSQL} % (fold)
\label{sub:postgresql}
% subsection posgresql (end)

\subsection{Groovlet} % (fold)
\label{sub:groovlet}
% subsection groovlet (end)

\subsection{Tomcat} % (fold)
\label{sub:tomcat}
% subsection tomcat (end)

\subsection{jQuery, HTML, CSS} % (fold)
\label{sub:jquery}
% subsection jquery (end)

\subsection{Maven} % (fold)
\label{sub:maven}
% subsection maven (end)

% section technologies_utilisées (end)

\section{Service métier} % (fold)
\label{sec:service_metier}
% section service_métier (end)

\section{Coordonateur de services} % (fold)
\label{sec:coordonateur_de_services}
% section coordonateur_de_services (end)

\section{Présentation web} % (fold)
\label{sec:presentation_web}
% section présentation_web (end)

\end{document}